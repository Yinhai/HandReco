\documentclass[11pt]{article}
\usepackage[utf8]{inputenc}
\usepackage{amsmath, amsthm, amssymb}
\usepackage{graphicx}
\author{Fabian Alenius, Kjell Winblad and Chongyuang Sun} \title{Handwritten Character Recgonition}
\begin{document}
\maketitle

\begin{abstract}
Hello.

\end{abstract}

\section{Introduction}
1. Give general introduction to area.
2. Describe why it's important.
3. Applications.
4. Contributions.

\cite{trec}
Fabian
testing 123.

\section{Previous work}
Write about other papers and how they have solved the problem.

\section{Problem}
Describe our limited version of the problem.
Describe the two problems, one is recognizing characters and the other is to recognize words.
1. Images have strokes of width 1.
2. Assume characters segment in word.

\section{Method}\label{sec:method}
2. General overview of our implementation.   Kjell
	Two classifiers
		What they are doing
3. Add picture describing implementation.  Kjell

1. Describe HMM, short overview. Chongyang
1 Picture of HMM topology
4. Describe the different initialization algos, advantages and disadvantages. Chongyang
1.1Topology of HMM Chongyang
1.2 prevention of underflow. Chongyang
1.3 Handle zeros in denominator. Chongyang 




\subsection{Dataset}\label{sec:dataset}
1. Write about creation of sample data. Chongyang
2. Could not find good dataset. Chongyang
3. Write down how much data we had. Chongyang

\subsection{Preprocessing}
1. Write about scaling of picture. Kjell
2. Segmentation of picture. Kjell

\section{Result}\label{sec:result}
1. Maybe result from character test using different initialization algos.
2. Basic results from testing the implementation
3. Depending on how much other results we have, discuss experience of playing around with the gui.

            


\subsection{Word Classifier and Different Initialization Methods}


In this section classification results for the words in table~\ref{tab:words_supported_by_classifier} will be presented. See section~\ref{sec:method}, for a description of the implementation of the classifiers. The training and test example words were randomly generated with the a generator using the properties in table~\ref{tab:word_generator_properties}. See section~\ref{sec:dataset}, for more information about the word example generator. To test the accuracy of the created classifiers, 100 test examples were used. Five test examples for every words. The test examples were generated using the same properties as the training examples. The two initialization methods count based initialization and random initialization were tested with 100, 200, 400, 800 and 1600 training examples. The results of the test is presented in table~\ref{tab:word_classifier_results_generated_data}.

\begin{table}[htb]
  \begin{center}
  \begin{tabular}{ l l l l l }
    dog      & cat       & pig     & love       & hate  \\
    scala    & python    & summer  & winter     & night  \\ 
    daydream & nightmare & animal  & happiness  & sadness \\ 
    tennis   & feminism  & fascism & socialism  & capitalism \\
  \end{tabular}
\end{center}
\caption{Words supported by the resulting classifier.} 
\label{tab:words_supported_by_classifier} 
\end{table}

\begin{table}[htb]
  \begin{center}
  \begin{tabular}{ l l }
    Probability of extra letter at position         & 0.03 \\
    Probability of extra letter missing at position & 0.7 \\ 
    Probability of wrong letter at position         & 0.1 \\ 
    Probability of letter missing at position       & 0.03 \\
  \end{tabular}
\end{center}
\caption{Properties obeyed by the word training example generator.} 
\label{tab:word_generator_properties} 
\end{table}

\begin{table}[htb]
  \begin{center}
  \begin{tabular}{ l l l l l l l }
    NOE    & RIBF   & CBIBT  & RIAT    & CBIAT  & RITT & CBITT \\ \hline
    $100$  & $0.03$ & $0.99$ & $0.01$  & $0.0$  & $6$  & $3$\\ 
    $200$  & $0.02$ & $1.0$  & $0.13$  & $0.36$ & $12$ & $5$\\ 
    $400$  & $0.02$ & $1.0$  & $0.9$   & $0.95$ & $23$ & $7$\\
    $800$  & $0.01$ & $1.0$  & $1.0$   & $1.0$  & $46$ & $13$\\   
    $1600$ & $0.05$ & $1.0$  & $1.0$   & $1.0$  & $96$ & $26$\\  
  \end{tabular}
\end{center}
\caption{Test with different number of training examples and different initialization methods.
	 NOE=''number of training examples for every word'',
         RIBF=''random initialization score before training'',
         CBIBT=''count based initialization score before training'',
         RIAT=''random initialization score after training'',
         CBIAT=''count based initialization score after training'',
         RITT=''random initialization training time (minutes)'',
         CBITT=''count based initialization training time (minutes)''} 
\label{tab:word_classifier_results_generated_data} 
\end{table}

\subsection{Character Classification With Different Parameters}

\begin{table}[htb]
  \begin{center}
  \begin{tabular}{ l | l l l l l l l l l }
    CCF/NOS & 4   & 5    & 6    & 7    & 8    & 9    & 10   & 11 & 12 \\ \hline
    $0.7$  & $30$ & $40$ & $41$ & $37$ & $45$ & $44$ & $48$ & $na$ & $na$\\ 
    $1.0$  & $35$ & $38$ & $39$ & $49$ & $45$ & $44$ & $47$ & $na$ & $na$\\ 
    $1.3$  & $40$ & $45$ & $49$ & $57$ & $57$ & $59$ & $53$ & $na$ & $na$\\
    $1.6$  & $42$ & $49$ & $55$ & $57$ & $56$ & $57$ & $54$ & $na$ & $na$\\   
    $1.9$  & $47$ & $55$ & $52$ & $62$ & $58$ & $59$ & $57$ & $na$ & $na$\\  
    $2.2$  & $50$ & $58$ & $57$ & $65$ & $65$ & $59$ & $62$ & $na$ & $na$\\ 
    $2.5$  & $52$ & $59$ & $61$ & $60$ & $61$ & $63$ & $65$ & $na$ & $na$\\ 
    $2.8$  & $60$ & $60$ & $63$ & $60$ & $60$ & $62$ & $62$ & $na$ & $na$\\ 
    $3.1$  & $na$ & $59$ & $53$ & $64$ & $65$ & $63$ & $65$ & $na$ & $na$\\ 
    $3.4$  & $na$ & $60$ & $62$ & $65$ & $63$ & $67$ & $69$ & $na$ & $na$\\ 
    $3.7$  & $na$ & $63$ & $59$ & $58$ & $63$ & $67$ & $67$ & $na$ & $na$\\ 
  \end{tabular}
\end{center}
\caption{Results for character classification test with different parameters. The scores are percentage of correctly classified characters.
	 NOS=''number of segments'',
         CCF=''component classification factor''}
\label{tab:word_classifier_results_generated_data} 
\end{table}


\section{Discussion}
1. Discuss the results.
2. Discuss importance of amount of training data and the effects on performance.

\section{Future Work}
1. 

\bibliographystyle{plain}	% (uses file "plain.bst")
\bibliography{myrefs}		% expects file "myrefs.bib"

\appendix

\section{Reproduce Results}

\section{Testing Handwriting Recognition in Graphical User Interface}

\end{document}