\documentclass[11pt]{article}
\usepackage[utf8]{inputenc}
\usepackage{amsmath, amsthm, amssymb}
\usepackage{graphicx}
\author{Fabian Alenius, Kjell Winblad and Chongyuang Sun} \title{Handwritten Character Recgonition}
\begin{document}
\maketitle

\begin{abstract}
Hello.

\end{abstract}

\section{Introduction}
1. Give general introduction to area.
2. Describe why it's important.
3. Applications.
4. Contributions.

\cite{trec}
Fabian
testing 123.

\section{Previous work}
Write about other papers and how they have solved the problem.

\section{Problem}
Describe our limited version of the problem.
Describe the two problems, one is recognizing characters and the other is to recognize words.
1. Images have strokes of width 1.
2. Assume characters segment in word.

\section{Method}
2. General overview of our implementation.   Kjell
	Two classifiers
		What they are doing
3. Add picture describing implementation.  Kjell

1. Describe HMM, short overview. Chongyang
1 Picture of HMM topology
4. Describe the different initialization algos, advantages and disadvantages. Chongyang
1.1Topology of HMM Chongyang
1.2 prevention of underflow. Chongyang
1.3 Handle zeros in denominator. Chongyang 




\subsection{Dataset}\label{sec:dataset}
1. Write about creation of sample data. Chongyang
2. Could not find good dataset. Chongyang
3. Write down how much data we had. Chongyang

\subsection{Preprocessing}
1. Write about scaling of picture. Kjell
2. Segmentation of picture. Kjell

\section{Result}\label{sec:result}
1. Maybe result from character test using different initialization algos.
2. Basic results from testing the implementation
3. Depending on how much other results we have, discuss experience of playing around with the gui.

\section{Discussion}
1. Discuss the results.
2. Discuss importance of amount of training data and the effects on performance.

\section{Future Work}
1. 

\bibliographystyle{plain}	% (uses file "plain.bst")
\bibliography{myrefs}		% expects file "myrefs.bib"
\end{document}